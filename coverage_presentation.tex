\documentclass{beamer}
\hypersetup{colorlinks=true,linkcolor=red}
\usetheme{Luebeck}
\useoutertheme[subsection=false]{smoothbars}
\useoutertheme{umbcfootline}

% Audience:      master students at computer science and engineering department
% Time budget:   30 min talk + 15 min questions + 15 min break
% Sructure:      start from general, go into details, leave 2-3 min for conclusions
% Thesis title:  Code coverage criteria and their effect on test suite qualities
% Goal:          Present our work, i.e. what we have done

% presentation metadata
\title{Code coverage criteria and their effect on test suite qualities}
\author[M.Kalkov \and D.Pamakha]
{\texorpdfstring
  {\begin{columns}
     \column{.45\linewidth}
     \centering
     Mikhail Kalkov\\
     \href{mailto:mikhail.kalkov@gmail.com}{\texttt{\small mikhail.kalkov@gmail.com}}
     \column{.45\linewidth}
     \centering
     Dzmitry Pamakha\\
     \href{mailto:pomaxodv@gmail.com}{\texttt{\small pomaxodv@gmail.com}}
   \end{columns}}
  {Mikhail Kalkov \and Dzmitry Pamakha}
}
\institute[Chalmers University of Technology]{
  Master Programme in Software Engineering and Technology\\
  Computer Science and Engineering Department\\
  Chalmers University of Technology\\
  Gothenburg, Sweden
}
\date[December 2013]{December 17, 2013}

\begin{document}

%--- the titlepage frame -------------------------%
\begin{frame}[plain]
  \titlepage
\end{frame}

%--- the contents frame --------------------------%
\frame{
  \frametitle{Contents}

\tableofcontents
}

%--- the presentation begins here ----------------%

%--------------------
\section{Introduction}


\begin{frame}
  \frametitle{The Internet}

\begin{itemize}
  \item \textbf{February 1958}. DARPA is established.
  \item \textbf{May 1961}. L. Kleinrock "Information Flow in Large Communication Nets"
  \item \textbf{October, 29 1969}. First ARPANET message
  % They meant to send "LOGIN", but only "LO" got through.
  \item \textbf{September 1971}. 18 nodes
  \item \textbf{March 1977}. 54 nodes
\end{itemize}
\end{frame}


\begin{frame}
  \frametitle{The \emph{Internet} Protocol}

1974. Vint Cerf and Bob Kahn develop the first version of the IP.\\
1981. The ARPANET switches to the Internet Protocol, version 4.\\

\medskip
Who could had thought that we would still be using IPv4 in 2009? 

\pause
\begin{center}Nobody could.\end{center}
\begin{quote}
"We thought we were doing an experiment to prove the technology and that if it worked we'd have an opportunity to do a production version of it. Well — it just escaped!" - Vint Cerf in 2008 (\href{http://www.youtube.com/watch?v=mZo69JQoLb8\#t=13m00s}{watch it}).
\end{quote}
\end{frame}


\begin{frame}
  \frametitle{IPv4 limitations}

\begin{itemize}
  \item \(2^{32} \approx 4,3 * 10^9\) addresses
  \item It scales badly
  \item The private address space is an afterthought
  \item Security is an afterthought
  \item Poor support for QoS provision and traffic engineering
\end{itemize}
\end{frame}


%--------------------
\section{Technical novations of IPv6}

\begin{frame}
  \frametitle{Scaling}

\begin{description}
  \item [128-bit addresses]
  \item allow further growth of the Internet, and have numerous far-reaching consequences.
  \pause
  \item [Fixed-length header]
  \item would decrease the load on router processors.
  \pause
  \item [No path fragmentation]
  \item also contributes to decreasing router load.
\end{description} 
\end{frame}


\begin{frame}
  \frametitle{Speed}

\begin{description}
  \item [Flow labeling]
  \item makes reservation of bandwidth trivial.
  \pause
  \item [Packet sizes (min and max)]
  \item were increased to suit modern networks.
\end{description}
\end{frame}


\begin{frame}
  \frametitle{Ease of use}

\begin{description}
  \item [Neighbor discovery]
  \item simplifies network management.
  \pause
  \item [Autoconfiguration]
  \item makes installation of new devices, and renumbering trivial.
  \pause
  \item [Improved support for multicasting]
  \item allows for more effective and secure bandwidth utilization.
\end{description}
\end{frame}

%--------------------
\section{The paradigm shift}

\begin{frame}
  \frametitle{"IPv4 with more addresses"}

\begin{itemize}
  \item The shift from address scarity to address abundance
  \pause
  \item Restoration of end-to-end connectivity (J. Postel, 1972)
\end{itemize}
\end{frame}

%--------------------
\section{Practical aspects}

\subsection{Basics of IPv6}

\begin{frame}
  \frametitle{Basics of IPv6}

\begin{description}
  \item [Sample IPv6 address]
  \item \texttt{2001:0DB8:7F61:88E6:FF00:89E6:5CC6:CE27}
  \item [A more common example]
  \item \texttt{2001:DB8::8:800:200C:417A}
  \item [IPv6 localhost]
  \item \texttt{::1}
\end{description}
\end{frame}

\subsection{Accomplishments}

\begin{frame}
  \frametitle{Accomplishments}

\begin{itemize}
  \item IPv6 connectivity was obtained for the RAMK Cisco Networking Laboratory.\\
  \item IPv6 was studied, and its understanding was captured in the thesis.\\
  \item Several applications were tested in IPv6 environments, and common pitfalls were identified.
\end{itemize}
\end{frame}

%--------------------
\section{Question time}

\begin{frame}
  \frametitle{The End}

\begin{center}
  Any questions?
\end{center}
\end{frame}

\end{document}

